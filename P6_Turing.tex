
\documentclass[12pt, a4paper]{article}
\usepackage{hyperref}
\hypersetup{
  colorlinks=true,
  linkcolor=blue,
  urlcolor=cyan,
}
\urlstyle{same}
\usepackage[utf8]{inputenc}
\usepackage{amsmath}
\usepackage{amsfonts}
\usepackage{amssymb}
\usepackage{graphicx}


\newtheorem{theorem}{Teorema.}
\newtheorem{lemma}[theorem]{Lema.}
\newtheorem{corollary}[theorem]{Corolario.}
\newtheorem{definition}[theorem]{Definici\'on:}
\newtheorem{example}[theorem]{Ejemplo:}
\newtheorem{problema}[theorem]{Problema:}
\newtheorem{remark}[theorem]{Observaci\'on:}

\usepackage{graphicx}
\usepackage[spanish]{babel}
%\usetheme{default}

\newcommand{\pp}{\mathbb{P}}
\newcommand{\zz}{\mathbb{Z}}
\newcommand{\rr}{\mathbb{R}}
\newcommand{\qq}{\mathbb{Q}}

\usepackage{tikz, tikz-3dplot}

\definecolor{cof}{RGB}{219,144,71}
\definecolor{pur}{RGB}{186,146,162}
\definecolor{greeo}{RGB}{91,173,69}
\definecolor{greet}{RGB}{52,111,72}
\date{}


\begin{document}
\title{PRÁCTICO 6 LENGUAJES FORMALES: Máquinas de Turing}
\author{Mauricio Velasco}
\maketitle{}
\begin{enumerate}
\item \begin{enumerate}
\item Diseñe y escriba una máquina de Turing que escanea hacia la derecha hasta que encuentra dos $a$'s consecutivas y luego se detiene. El alfabeto de la máquina debe ser $\Sigma =\{a,b,\cup,\triangle\}$ y debe dar la descripción de la máquina en completo detalle (como tupla).
\item Escriba las configuraciones que ocurren al ejecutar su máquina con input $\cup bbabaa$.
\end{enumerate}

\item Construya una máquina de Turing (usando nuestra notación abreviada) que calcule la funcion $f: \{a,b\}^*\rightarrow \{a,b\}^*$ dada por $f(w)=ww^R$ donde $w^R$ significa la palabra reversa a $w$. Muestre la ejecución de la misma en una cadena representativa.


\item Describa una máquina de Turing que semidecida el lenguaje $a^*ba^*b$.

\item (\emph{Autómatas con dos stacks}) \begin{enumerate}
\item Defina formalmente un automata que sea un \emph{pushdown automata con dos stacks}, especificando definición, configuraciones y cómputo. Defina formalmente lo que significa que esta máquina \emph{acepte} un lenguaje.
\item Demuestre que un lenguaje es recursivo si y solo si es aceptado por un \emph{pushdown automata con dos stacks}.
\end{enumerate}
\item Encuentre gramáticas que generen los siguientes lenguajes:
\begin{enumerate}
\item $L=\{ww: w\in \{a,b\}^*\}$
\item $L=\{a^{2^n}: n\in\mathbb{N}\}$
\item $L=\{a^{n^2}: n\in\mathbb{N}\}$
\end{enumerate}

\end{enumerate}


\end{enumerate}




\end{document}