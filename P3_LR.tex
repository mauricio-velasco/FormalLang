
\documentclass[12pt, a4paper]{article}
\usepackage{hyperref}
\hypersetup{
  colorlinks=true,
  linkcolor=blue,
  urlcolor=cyan,
}
\urlstyle{same}
\usepackage[utf8]{inputenc}
\usepackage{amsmath}
\usepackage{amsfonts}
\usepackage{amssymb}
\usepackage{graphicx}


\newtheorem{theorem}{Teorema.}
\newtheorem{lemma}[theorem]{Lema.}
\newtheorem{corollary}[theorem]{Corolario.}
\newtheorem{definition}[theorem]{Definici\'on:}
\newtheorem{example}[theorem]{Ejemplo:}
\newtheorem{problema}[theorem]{Problema:}
\newtheorem{remark}[theorem]{Observaci\'on:}

\usepackage{graphicx}
\usepackage[spanish]{babel}
%\usetheme{default}

\newcommand{\pp}{\mathbb{P}}
\newcommand{\zz}{\mathbb{Z}}
\newcommand{\rr}{\mathbb{R}}
\newcommand{\qq}{\mathbb{Q}}

\usepackage{tikz, tikz-3dplot}

\definecolor{cof}{RGB}{219,144,71}
\definecolor{pur}{RGB}{186,146,162}
\definecolor{greeo}{RGB}{91,173,69}
\definecolor{greet}{RGB}{52,111,72}
\date{}


\begin{document}
\title{PRÁCTICO 3 LENGUAJES FORMALES: GNFAs y el pumping Lemma.}
\author{Mauricio Velasco}
\maketitle{}

\begin{enumerate}
\item Utilice el pumping lemma para demostrar que los siguientes lenguajes NO SON regulares. Incluya una argumentación clara.
\begin{enumerate}
\item $A_1=\{0^n1^n2^n:n\in \mathbb{N}\}$ 
\item $A_2=\{www|w\in \{a,b\}^*\}$
\item $A_3=\{a^{2^n}:n\in \mathbb{N}\}$
\end{enumerate}

\item Para un string $w_1\dots w_n$ la reversa es $w_nw_{n-1}\dots w_2w_1$. Demuestre que un lenguaje $A$ es regular si y solo si el conjunto de sus reversas es tambien regular.

\item Demuestre que los siguientes lenguajes no son regulares:
\begin{enumerate}
\item $\{0^m1^n: m\neq n\}$
\item $\{w\in \{0,1\}^* \text{que NO son palabras capicua (a.k.a. palindromas)}\}$
\end{enumerate}

\item Escriba formalmente las siguientes:
\begin{enumerate}
\item Definición de GNFA.
\item Complete: El GNFA $M$ acepta la palabra $w$ si...
\item Complete: Para todo GNFA existe una expresión regular tal que...
\item De un ejemplo de un GNFA con tres estados y construya la expresión regular del lenguaje que este acepta, obtenida eliminando el estado que no es de inicio o aceptación.
\end{enumerate}

\item Un {\bf TODO-NFA} es una $5$-tupla $(Q,\Sigma,\delta,q_0,F)$ que acepta $w^*$ si \emph{todo} estado al que puede llegar despues de leer $w^*$ esta en $F$. Demuestre que los {\bf TODO-NFAs} tambien reconocen la colección de lenguajes regulares.

\item Sea $L\subseteq \Sigma^*$ un lenguaje. Decimos que $x$ y $y$ puden distinguirse mediante $L$ si existe un $z\in \Sigma^*$ tal que ex\áctamente uno de $xz$ y $yz$ estan en $L$. Para $x,y\in \Sigma^*$ definimos $x\sim y$ si $x$ y $y$ NO PUEDEN distinguirse mediante $L$. Demuestre que $\sim$ es una relación de equivalencia (es decir es reflexiva, simétrica y transitiva).



\end{document}