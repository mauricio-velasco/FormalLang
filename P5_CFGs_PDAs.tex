
\documentclass[12pt, a4paper]{article}
\usepackage{hyperref}
\hypersetup{
  colorlinks=true,
  linkcolor=blue,
  urlcolor=cyan,
}
\urlstyle{same}
\usepackage[utf8]{inputenc}
\usepackage{amsmath}
\usepackage{amsfonts}
\usepackage{amssymb}
\usepackage{graphicx}


\newtheorem{theorem}{Teorema.}
\newtheorem{lemma}[theorem]{Lema.}
\newtheorem{corollary}[theorem]{Corolario.}
\newtheorem{definition}[theorem]{Definici\'on:}
\newtheorem{example}[theorem]{Ejemplo:}
\newtheorem{problema}[theorem]{Problema:}
\newtheorem{remark}[theorem]{Observaci\'on:}

\usepackage{graphicx}
\usepackage[spanish]{babel}
%\usetheme{default}

\newcommand{\pp}{\mathbb{P}}
\newcommand{\zz}{\mathbb{Z}}
\newcommand{\rr}{\mathbb{R}}
\newcommand{\qq}{\mathbb{Q}}

\usepackage{tikz, tikz-3dplot}

\definecolor{cof}{RGB}{219,144,71}
\definecolor{pur}{RGB}{186,146,162}
\definecolor{greeo}{RGB}{91,173,69}
\definecolor{greet}{RGB}{52,111,72}
\date{}


\begin{document}
\title{PRÁCTICO 5 LENGUAJES FORMALES: Grámaticas libres de contexto (CFGs) y Automatas de pila (PDAs)}
\author{Mauricio Velasco}
\maketitle{}

\begin{enumerate} 

\item Sea $\Sigma=\{a,b,(,),\cup,*,\emptyset\}$. Construya un CFG que genere todas las cadenas de $\Sigma^*$ que sean expresiones regulares sobre $\{a,b\}$.

\item Sea $V=\{a,b,S,A,B\}$, $\Sigma =\{a,b\}$ y defina $R$ mediante las reglas 
\[
\begin{cases}
S\rightarrow aB\\
S\rightarrow bA\\
A\rightarrow a\\
A\rightarrow aS\\
A\rightarrow BAA\\
B\rightarrow b\\
B\rightarrow bS\\
B\rightarrow ABB\\
\end{cases}
\]
\begin{enumerate}
\item Demuestre que $ababba\in L(G)$ 
\item Muestre que $L(G)$ es el conjunto de todas las cadenas no vacias que tienen la misma cantidad de ocurrencias de $a$ y de $b$.
\end{enumerate}

\item Demuestren que el conjunto de lenguajes generados por algun CFG (o equiv. aceptados por algun PDA) cumple:
\begin{enumerate}
\item Es cerrada bajo las operaciones de unión, concatenación y estrella de Kleene.
\item No es cerrada ni bajo intersección ni bajo complementos (Nota: Esto requiere construir contraejemplos explícitos y usar alguna versión del pumping lemma para CFGs).
\end{enumerate}




\item Considere el PDA $M$ con
$K=\{s,f\}$, $\Sigma=\{a,b\}$, $\Gamma=\{a\}$ y con $\Delta$ dado por  
\[\{[(s,a,e),(s,a)], [(s,b,e),(s,a)],[(s,a,e),(f,e)],[(f,a,a),(f,e)],[(f,b,a),(f,e)]\}\]
\begin{enumerate}
\item Escriba todas las posibles sucesiones de transiciones de $M$ con input $aba$.
\item Demuestre que $aba,aa,bb\not\in  L(M)$ pero $baa,bab,baaa\in L(M)$.
\item Describa $L(M)$ en palabras. 
\end{enumerate}
\item Construya PDAs que acepten cada uno de los siguientes lenguajes del alfabeto $\Sigma=\{a,b\}$:
\begin{enumerate}
\item El lenguaje $\{a^mb^n: m\leq n\leq 2m, m\in \mathbb{N}\}$
\item El lenguaje $\{w\in \Sigma^*: w=w^R\}$ donde $w^R$ denota la palabra reversa.
\item El lenguaje 
\[\left\{w\in \Sigma^*:\text{$w$ tiene el doble de $b$'s que de $a$'s}\right\}\]
\end{enumerate}

\item Sea $M$ un PDA. El \emph{lenguaje aceptado por $M$ con estado final $f_0\in F$} se define como
\[L_{f_0}(M):=\{w\in \Sigma^*: (s,w,\epsilon)\vdash (f_0,\epsilon,\alpha)\text{ para algún $\alpha\in \Gamma^*$}\}\]
Demuestre que existe un PDA $M'$ con $L(M')=L(M_{f_0})$.


\end{document}