
\documentclass[12pt, a4paper]{article}
\usepackage{hyperref}
\hypersetup{
  colorlinks=true,
  linkcolor=blue,
  urlcolor=cyan,
}
\urlstyle{same}
\usepackage[utf8]{inputenc}
\usepackage{amsmath}
\usepackage{amsfonts}
\usepackage{amssymb}
\usepackage{graphicx}


\newtheorem{theorem}{Teorema.}
\newtheorem{lemma}[theorem]{Lema.}
\newtheorem{corollary}[theorem]{Corolario.}
\newtheorem{definition}[theorem]{Definici\'on:}
\newtheorem{example}[theorem]{Ejemplo:}
\newtheorem{problema}[theorem]{Problema:}
\newtheorem{remark}[theorem]{Observaci\'on:}

\usepackage{graphicx}
\usepackage[spanish]{babel}
%\usetheme{default}

\newcommand{\pp}{\mathbb{P}}
\newcommand{\zz}{\mathbb{Z}}
\newcommand{\rr}{\mathbb{R}}
\newcommand{\qq}{\mathbb{Q}}

\usepackage{tikz, tikz-3dplot}

\definecolor{cof}{RGB}{219,144,71}
\definecolor{pur}{RGB}{186,146,162}
\definecolor{greeo}{RGB}{91,173,69}
\definecolor{greet}{RGB}{52,111,72}

\date{}

\begin{document}
\title{PRÁCTICO 2 LENGUAJES FORMALES: Automatas finitos no-deterministas (AFNDs) y expresiones regulares.}
\author{Mauricio Velasco}
\maketitle{}

\begin{enumerate}
\item Sea $N$ un AFND y sea $w=y_1y_2\dots y_m$ una palabra. Complete de manera precisa la siguiente definición: $N$ \emph{acepta} a $w$ si... 

\item Construya AFNDs con el número especificado de estados que reconozcan los siguientes lenguajes (usando el alfabeto $\{0,1\}$):
\begin{enumerate}
\item El lenguaje constituido por las palabras que terminan en $00$ usando TRES estados.
\item El lenguaje $0^*1^*0^*$ usando TRES estados.
\item El lenguaje constituido por las palabras que contienen al substring $\verb!0101!$ usando  CINCO estados.
\item El lenguaje $0^*$ con UN estado.
\end{enumerate}

\item Defina expresiones regulares que generen los siguientes lenguajes en el alfabeto $\{0,1\}$:

\begin{enumerate}
\item Las palabras que empiezan con $1$ y terminan con $0$.
\item Las palabras que contienen al menos tres unos.
\item Las palabras de longitud a lo más cinco.
\item Las palabras en las que toda posición impar es un uno.
\item Las palabras que ó contienen un número par de ceros ó exáctamente dos unos.
\end{enumerate}

\item Realice las siguientes tareas:
\begin{enumerate}
\item Construya un AFND que reconozca el lenguaje $\left(01\cup 001\cup 010\right)^*$.
\item Convierta esta AFND en un AFD equivalente siguiendo la prueba de la equivalencia vista en clase.
\end{enumerate}

\item Convierta las siguientes expresiones regulares en AFND:
\begin{enumerate}
\item $(0\cup 1)^*000(0\cup 1)^*$ 
\item $\emptyset^*$
\end{enumerate}

\item Sea $B$ un lenguaje cualquiera sobre el alfabeto $\Sigma$. Demuestre que $B=B^*$ si y sólo si $BB=B$. 


\end{enumerate}



\end{enumerate}
\end{document}