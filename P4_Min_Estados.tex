
\documentclass[12pt, a4paper]{article}
\usepackage{hyperref}
\hypersetup{
  colorlinks=true,
  linkcolor=blue,
  urlcolor=cyan,
}
\urlstyle{same}
\usepackage[utf8]{inputenc}
\usepackage{amsmath}
\usepackage{amsfonts}
\usepackage{amssymb}
\usepackage{graphicx}


\newtheorem{theorem}{Teorema.}
\newtheorem{lemma}[theorem]{Lema.}
\newtheorem{corollary}[theorem]{Corolario.}
\newtheorem{definition}[theorem]{Definici\'on:}
\newtheorem{example}[theorem]{Ejemplo:}
\newtheorem{problema}[theorem]{Problema:}
\newtheorem{remark}[theorem]{Observaci\'on:}

\usepackage{graphicx}
\usepackage[spanish]{babel}
%\usetheme{default}

\newcommand{\pp}{\mathbb{P}}
\newcommand{\zz}{\mathbb{Z}}
\newcommand{\rr}{\mathbb{R}}
\newcommand{\qq}{\mathbb{Q}}

\usepackage{tikz, tikz-3dplot}

\definecolor{cof}{RGB}{219,144,71}
\definecolor{pur}{RGB}{186,146,162}
\definecolor{greeo}{RGB}{91,173,69}
\definecolor{greet}{RGB}{52,111,72}
\date{}


\begin{document}
\title{PRÁCTICO 4 LENGUAJES FORMALES: Minimización de estados.}
\author{Mauricio Velasco}
\maketitle{}

\begin{enumerate} 
\item Sea $\Sigma =\{a,b\}$. Encuentre las clases de equivalencia de la relación $\simeq_L$ para los siguientes lenguajes $L\subseteq \Sigma^*$:
\begin{enumerate}
\item $L=\{x\in \Sigma^*: x\text{ contiene al menos una ocurrencia de aababa}\}$
\item $L=\{xx:x\in \Sigma^*\}$
\end{enumerate}
\item Para cada uno de los lenguajes del ejercicio anterior encuentre el automata óptimo que acepta a $L$.

\item Una palabra $z\in \Sigma^*$ se llama \emph{libre de cuadrados} si NO puede escribirse como $z=uvvw$ para $u,v,w\in \Sigma^*$ con $v\neq \epsilon$. (por ejemplo $perro$ y $papa$ no son palabras libres de cuadrados). Demuestre que si $|\Sigma|\geq 2$ entonces el lenguaje que consiste de las palabras libres de cuadrados en $\Sigma^*$ NO es un lenguaje regular.

\item Demuestre que las expresiones regulares $aa(a\cup b)^*\cup (bb)^*a^*$ y $(ab\cup ba\cup a)^*$ NO describen el mismo lenguaje de dos maneras. Aplicando sobre ellas un algoritmo general y encontrando una cadena generada por una y no por la otra.


 
\end{enumerate}


\end{document}